\documentclass{patmorin}
\listfiles
\usepackage[utf8]{inputenc}
\usepackage{microtype}
\usepackage{amsthm,amsmath,graphicx}
\usepackage{pat}
\usepackage[letterpaper]{hyperref}
\usepackage[table,dvipsnames]{xcolor}
\definecolor{linkblue}{named}{Blue}
\hypersetup{colorlinks=true, linkcolor=linkblue,  anchorcolor=linkblue,
citecolor=linkblue, filecolor=linkblue, menucolor=linkblue,
urlcolor=linkblue} 
\setlength{\parskip}{1ex}
\usepackage{wasysym}

\title{\MakeUppercase{Anagram-Free Chromatic Number is not Pathwidth Bounded}}

\author{Pat Morin and Friends}%

%\usepackage[mathlines]{lineno}
%\linenumbers
%\setlength{\linenumbersep}{2.5cm}
%\rightlinenumbers
%\linenumbers
%\newcommand*\patchAmsMathEnvironmentForLineno[1]{%
%  \expandafter\let\csname old#1\expandafter\endcsname\csname #1\endcsname
%  \expandafter\let\csname oldend#1\expandafter\endcsname\csname end#1\endcsname
%  \renewenvironment{#1}%
%     {\linenomath\csname old#1\endcsname}%
%     {\csname oldend#1\endcsname\endlinenomath}}% 
%\newcommand*\patchBothAmsMathEnvironmentsForLineno[1]{%
%  \patchAmsMathEnvironmentForLineno{#1}%
%  \patchAmsMathEnvironmentForLineno{#1*}}%
%\AtBeginDocument{%
%\patchBothAmsMathEnvironmentsForLineno{equation}%
%\patchBothAmsMathEnvironmentsForLineno{align}%
%\patchBothAmsMathEnvironmentsForLineno{flalign}%
%\patchBothAmsMathEnvironmentsForLineno{alignat}%
%\patchBothAmsMathEnvironmentsForLineno{gather}%
%\patchBothAmsMathEnvironmentsForLineno{multline}%
%}


\newcommand{\question}[1]{\textbf{\color{red}Question:}~#1}

\DeclareMathOperator{\ob}{obs}
\DeclareMathOperator{\planeobs}{plane-obs}

\newcommand{\eps}{\epsilon}

%\pagenumbering{roman}
\begin{document}
%\begin{titlepage}
\maketitle
%
\begin{abstract}
\end{abstract}
%\end{titlepage}
%
%\tableofcontents
%
%\newpage


\section{Introduction}
\pagenumbering{arabic}

For a string $s$ over an alphabet $\Sigma$, $n_i(s)$ denotes the number
of occurrences of $i\in\Sigma $ in $s$.  We say that a string $p=st$ with
$|s|=|t|$ is \emph{balanced} if, for all $i\in\Sigma$, $n_i(s)\le 2n_i(t)$
and $n_i(t)\le 2n_i(s)$.




\begin{lem}
  Let $r$ and $h$ be natural numbers and let $n=2^h$. Then there
  exists a probability distribution $\mathcal{D}$ over $\{0,\ldots,n\}^2$ such that, for any binary string $s$ of length $n$ not containing $11$ or $0^r$,
  $\Pr_{(i,\ell)\sim\mathcal{D}}\{n_1(s_i\ldots,s_{i+\ell-1}) > 2n_1(s_{i+\ell},s_{i+2\ell-1} \} = o_h(1)$.
\end{lem}

\begin{lem}
  For any integer $c$ there exists an integer $n$ such
  that any string of length $n$ over an alphabet $\Sigma$ of size $c$
  contains a $k$-balanced substring.
\end{lem}

\begin{proof}
  Call a string \emph{good} if it contains no $k$-balanced substring.
  Suppose the lemma is not true so that there exists some minimum value
  $c$ for which there exist arbitrarily long good strings over an alphabet
  $\Sigma$ of size $c$.  

  Since $c$ is the minimum such value, this implies that there exists an
  integer $r$ such that every good string of length $r$ uses an alphabet
  of size at least $c$.  This implies that, in any good string over
  $\Sigma$, any substring of length $r$ contains at least one occurrence
  of each element of $\Sigma$.

  For any $i\in\Sigma$ and any $j\in\{1,\ldots,n-2r+1\}$, let
  $L_{i,j}=n_i(s_{j}\ldots s_{j+r-1})$ and let $R_{i,j}=n_i(s_{j+r}\ldots
  s_{j+2r-1})$.  That is, $L_{i,j}$ and $R_{i,j}$ count occurrences of $i$
  in the first and second half of the string $s_{j}\ldots s_{j+2r-1}$. In the remainder of this proof, all probability and expectations are taken with respect to a uniformly random choice of $i\in\{1,\ldots,n-2r+1\}$.

  First, we claim that
  \begin{equation}
    
  \end{equation}

\end{proof}


\end{document}


