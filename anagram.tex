\documentclass{patmorin}
\listfiles
\usepackage[utf8]{inputenc}
\usepackage{microtype}
\usepackage{amsthm,amsmath,graphicx}
\usepackage{pat}
\usepackage[letterpaper]{hyperref}
\usepackage[table,dvipsnames]{xcolor}
\definecolor{linkblue}{named}{Blue}
\hypersetup{colorlinks=true, linkcolor=linkblue,  anchorcolor=linkblue,
citecolor=linkblue, filecolor=linkblue, menucolor=linkblue,
urlcolor=linkblue} 
\setlength{\parskip}{1ex}
\usepackage{wasysym}

\title{\MakeUppercase{Anagram-Free Chromatic Number is not Pathwidth Bounded}}

\author{Pat Morin and Friends}%

%\usepackage[mathlines]{lineno}
%\linenumbers
%\setlength{\linenumbersep}{2.5cm}
%\rightlinenumbers
%\linenumbers
%\newcommand*\patchAmsMathEnvironmentForLineno[1]{%
%  \expandafter\let\csname old#1\expandafter\endcsname\csname #1\endcsname
%  \expandafter\let\csname oldend#1\expandafter\endcsname\csname end#1\endcsname
%  \renewenvironment{#1}%
%     {\linenomath\csname old#1\endcsname}%
%     {\csname oldend#1\endcsname\endlinenomath}}% 
%\newcommand*\patchBothAmsMathEnvironmentsForLineno[1]{%
%  \patchAmsMathEnvironmentForLineno{#1}%
%  \patchAmsMathEnvironmentForLineno{#1*}}%
%\AtBeginDocument{%
%\patchBothAmsMathEnvironmentsForLineno{equation}%
%\patchBothAmsMathEnvironmentsForLineno{align}%
%\patchBothAmsMathEnvironmentsForLineno{flalign}%
%\patchBothAmsMathEnvironmentsForLineno{alignat}%
%\patchBothAmsMathEnvironmentsForLineno{gather}%
%\patchBothAmsMathEnvironmentsForLineno{multline}%
%}


\newcommand{\question}[1]{\textbf{\color{red}Question:}~#1}

\DeclareMathOperator{\ob}{obs}
\DeclareMathOperator{\planeobs}{plane-obs}

\newcommand{\eps}{\epsilon}

%\pagenumbering{roman}
\begin{document}
%\begin{titlepage}
\maketitle
%
\begin{abstract}
\end{abstract}
%\end{titlepage}
%
%\tableofcontents
%
%\newpage


\section{Introduction}
\pagenumbering{arabic}

Let $s$ be a string over some alphabet $\Sigma$.  For each $i\in\Sigma$,
$n_i(s)$ denotes the number of occurences of $i$ in $s$.  If $s=pq$ is
the concatenation of two equal length strings $p$ and $q$, then we say
that $s$ is \emph{$i$-balanced} if $n_i(q)/2\le n_i(p)\le 2n_i(q)$.
If $s$ is $i$-balanced for each $i\in \Sigma$, then we say that $s$
is \emph{balanced}.

A binary string (over the alphabet $\Sigma=\{0,1\}$) is \emph{$r$-rich}
if it does not contain $0^r$ as a substring and it is \emph{light}
if it does not contain $11$ as a substring.


\begin{lem}\lemlabel{binary}
  Let $r$ and $h$ be natural numbers and let $s=s_0\ldots,s_{2^h-1}$ be any light $r$-rich
  binary string of length $2^h$.  If we choose $i\in\{0,\ldots,h-1\}$ uniformly at random and choose $j\in\{0,\ldots,2^{i}-1\}$ uniformly at random, then
  with probability $1-o_h(1)$, the substring $s_{j2^{h-i}}\ldots s_{(j+1)2^{h-i}-1}$ is 1-balanced.
\end{lem}

\begin{proof}
  This is the tricky bit.
\end{proof}

\begin{lem}\lemlabel{balanced}
  For any integer $c$ there exists an integer $n$ such that any string
  of length $n$ over an alphabet $\Sigma$ of size $c$ contains a balanced
  substring.
\end{lem}

\begin{proof}
  Call a string \emph{good} if it contains no balanced substring.
  Suppose the lemma is not true so that there exists some minimum value
  $c$ for which there exist arbitrarily long good strings over an alphabet
  $\Sigma$ of size $c$.  
  Since $c$ is the minimum such value, this implies that there exists an
  integer $r$ such that every good string of length $r$ uses an alphabet
  of size at least $c$.  This implies that, in any good string over
  $\Sigma$, any substring of length $r$ contains at least one occurrence
  of each element of $\Sigma$.

  Let $s$ be a good string of length $2^h$ and, for each $i\in\Sigma$,
  consider the binary string $s[i]$ derived from $s$ by replacing
  each occurrence of $i$ with a 1 and every other symbol with a 0.
  Observe that $s[i]$ is light since the appearance of the substring $11$
  in $s[i]$ would imply that $s$ contains the balanced substring $ii$.
  Observe also that $s[i]$ is $r$-rich since, from the discussion above,
  every substring of $s$ of length $r$ contains at least one occurrence
  of $i$.

  Now, choose $i\in\{0,\ldots,h-1\}$ uniformly at random and
  choose $j\in\{0,\ldots,2^{i}-1\}$ uniformly at random and let
  $t=s_{j2^{h-i}}\ldots s_{(j+1)2^{h-i}-1}$.  By \lemref{binary}, the
  probability that $t[i]$ is not 1-balanced is $o_h(1)$, which implies
  that the probability that $t$ is not $i$-balanced is $o_h(1)$.  Now,
  by the union bound, the probability that $t$ is not balanced is is
  at most $c\cdot o_h(1) < 1$ for sufficiently large $h$. We conclude
  that any sufficiently long string $s\in \Sigma^*$ must contain a
  balanced substring, contradicting that the assumption that there
  exist arbitrarily long strings over an alphabet of size $c$ having no
  balanced substrings.
\end{proof}


\begin{thm}
  For every integer $\alpha$, there exists a graph of pathwidth 3 whose
  non-repetitive chromatic number is greater than $\alpha$.
\end{thm}

\begin{proof}
Let $G$ be the graph with vertex set
$V(G)=\{v_0,\ldots,v_{n-1},w_0,\ldots,w_{n-1}\}$ and with edge set
\[
  E(G) = \{v_iw_i : i\in\{0,\ldots,n-1\}\} \cup \bigcup_{i=0}^{n-2} \{v_iw_{i+1},v_{i+1}w_i\} \enspace .
\]
We call the pairs $v_i$ and $w_i$ \emph{twins}.
The graph $G$ has pathwidth 3 as can be seen from the path decomposition $B_0,\ldots,B_{n-2}$ where $B_i=\{v_i,w_i,v_{i+1},w_{i+1}\}$.

Now, consider some colouring $\varphi:V(G)\to\{1,\ldots,\alpha\}$ of $G$
using $\alpha$ colours. From this we derive a string $s=s_0\ldots,s_{n-1}$
where $s_i=\{\varphi(v_i),\varphi(w_i)\}$.  Observe that $s$ is
a string over an alphabet $\Sigma$ of size $c=\binom{\alpha}{2}$.
By \lemref{balanced}, $s$ contains a balanced substring $t$ if $n$
is sufficiently large.  Since $t$
is balanced, a na\"ive greedy algorithm allows us to construct a bipartite
 graph $B$ whose edges join
elements of the first half of $t$ to equal elements in the second half
and having the following properties: $B$ has minimum degree 1, maximum
degree 2 and contains no path of length greater than 2.  Thus, the
connected components of $B$ are paths of length 1 and paths of length 2.

We will use $B$ to construct a path $P$ in $G$ whose colouring is an
anagram.  For each length-1 component $s_is_j$ of $B$, the path $P$
will include $v_i$, $w_i$, $v_j$ and $w_j$.  For each length-2 path
$s_is_js_k$ the path $P$ include $v_i$, $v_j$, $w_j$, and the unique
vertex $u\in\{v_k,w_k\}$ such that $\varphi(u)\neq \varphi(v_i)$.
The ordering of vertices along the path $P$ is by increasing order
of index, with ties (among $v_i$ and $w_i$) broken arbitrarily.
The fact that $B$ has minimum degree 1 ensures that $P=x_1,\ldots,x_m$
is indeed a path in $G$.  The choice of vertices in $P$ ensures that
$\varphi(x_1),\ldots,\varphi(x_m)$ is an anagram since length-1 components
of $B$ create twins in the first half that cancel twins in the second
half and length-2 components create twins in one half that cancel two
vertices in the other half.
\end{proof}

\end{document}


