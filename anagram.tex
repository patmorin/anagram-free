\documentclass{patmorin}
\listfiles
\usepackage[utf8]{inputenc}
\usepackage{microtype}
\usepackage{amsthm,amsmath,graphicx}
\usepackage{pat}
\usepackage[letterpaper]{hyperref}
\usepackage[table,dvipsnames]{xcolor}
\definecolor{linkblue}{named}{Blue}
\hypersetup{colorlinks=true, linkcolor=linkblue,  anchorcolor=linkblue,
citecolor=linkblue, filecolor=linkblue, menucolor=linkblue,
urlcolor=linkblue} 
\setlength{\parskip}{1ex}
\usepackage{wasysym}

\title{\MakeUppercase{Anagram-Free Chromatic Number is not Pathwidth Bounded}}

\author{Pat Morin and Friends}%

%\usepackage[mathlines]{lineno}
%\linenumbers
%\setlength{\linenumbersep}{2.5cm}
%\rightlinenumbers
%\linenumbers
%\newcommand*\patchAmsMathEnvironmentForLineno[1]{%
%  \expandafter\let\csname old#1\expandafter\endcsname\csname #1\endcsname
%  \expandafter\let\csname oldend#1\expandafter\endcsname\csname end#1\endcsname
%  \renewenvironment{#1}%
%     {\linenomath\csname old#1\endcsname}%
%     {\csname oldend#1\endcsname\endlinenomath}}% 
%\newcommand*\patchBothAmsMathEnvironmentsForLineno[1]{%
%  \patchAmsMathEnvironmentForLineno{#1}%
%  \patchAmsMathEnvironmentForLineno{#1*}}%
%\AtBeginDocument{%
%\patchBothAmsMathEnvironmentsForLineno{equation}%
%\patchBothAmsMathEnvironmentsForLineno{align}%
%\patchBothAmsMathEnvironmentsForLineno{flalign}%
%\patchBothAmsMathEnvironmentsForLineno{alignat}%
%\patchBothAmsMathEnvironmentsForLineno{gather}%
%\patchBothAmsMathEnvironmentsForLineno{multline}%
%}


\newcommand{\question}[1]{\textbf{\color{red}Question:}~#1}

\DeclareMathOperator{\ob}{obs}
\DeclareMathOperator{\planeobs}{plane-obs}

\newcommand{\eps}{\epsilon}

%\pagenumbering{roman}
\begin{document}
%\begin{titlepage}
\maketitle
%
\begin{abstract}
\end{abstract}
%\end{titlepage}
%
%\tableofcontents
%
%\newpage


\section{Introduction}
\pagenumbering{arabic}

Let $s$ be a string over some alphabet $\Sigma$.  For each $i\in\Sigma$,
$n_i(s)$ denotes the number of occurences of $i$ in $s$.  If $s=pq$ is the
concatenation of two equal length strings $p$ and $q$, then we say that
$s$ is \emph{$i$-balanced} if $n_i(q)/2\le n_i(p)\le 2n_i(q)$ otherwise
$s$ is $i$-unbalanced.  If $s$ is $i$-balanced for each $i\in \Sigma$,
then we say that $s$ is \emph{balanced}, otherwise $s$ is unbalanced.

\begin{lem}\lemlabel{balanced}
  For any integer $c$ there exists an integer $n$ such that any string
  of length $n$ over an alphabet $\Sigma$ of size $c$ contains a balanced
  substring.
\end{lem}

\begin{proof}
  Call a string a \emph{$c$-string} if it is over an alphabet of size $c$
  and call a string \emph{good} if it contains no balanced substring.  Let
  $h_c$ denote the minimum integer for which there is no good $c$-string
  length $2^{h_c}$.  Thus, our goal is to show that $h_c<\infty$.

  Suppose the lemma is not true so that there exists some minimum value
  $c$ for which $h_c$ is infinite.  Choosing the minimum such value of 
  $c$ ensures that $h_{c-1}$ is finite and that any good (sub)string of
  length at least $r:=2^{h_{c-1}}$ uses an alphabet of size at least $c$.

  Let $s$ be a good $c$-string of length $r2^{h}$, for some value $h$
  to specified shortly.  Assume that $s$ is a string over the alphabet
  $\Sigma=\{1,\ldots,c\}$. Consider the complete binary tree $T$ of
  height $h$ whose leaves, in order, are the length-1 substrings of
  $s$ and for which each internal node is the substring obtained by
  concatenating the node's left and right child.

  For each node $v$ of $T$, let $h(v)$ denote the height of $v$'s subtree
  and $s(v)=2^{h(v)}$ denote the length of the string $v$. For each
  $i\in\Sigma$, let $w_i(v):=n_i(v)/s(v)$.  Note that $0\le w_i(v)\le
  1/2$ and that $\sum_{i\in\Sigma} w_i(v)=1$.  Furthermore, if $v$
  has two children $x$ and $y$, then $w_i(v) = (w_i(x)+w_i(y))/2$.
  The assumption that $s$ has no balanced substring implies $v$ is
  $i$-unbalanced, for some $i\in\Sigma$.  Assign each internal node
  $v$ of $T$ the \emph{label} $\ell(v):=\min\{i\in\Sigma: \mbox{$v$
  is $i$-unbalanced}\}$.

  For $i\in\Sigma$, let $S_i=\{v\in V(T): \ell(v)=i\}$ and observe that 
  \[
      \sum_{i\in\Sigma} \sum_{v\in S_i} s(v) = (h+\log_2 r)r2^{h}
  \]
  and therefore, there exists some $i^*\in\Sigma$ such that
  $s(S_i)\ge (h+\log_2 r)r2^h/c$.  Let $X=S_{i^*}$ and $w=w_{i^*}$. From
  this point on we use the shorhands (for any $R\subseteq V(T)$, 
  $s(R):=\sum_{v\in R}s(v)$ and $w(R):=\sum_{v\in R}s(v)w(v)/s(R)$.

  Summarizing, we have a complete binary tree $T$ of height $h+\log_2 r$ and
  a subset $X\subseteq V(T)$ with the following properties: 
  \begin{enumerate}
    \item For each $v\in V(T)$, $w(v) \le 1/2$.
    \item For each $v\in V(T)$ with $s(v) \ge r$, $w(v)\ge 1/r$.
    \item For each internal node $v\in V(T)$ with children $x$ and $y$,
       $w(v) = (w(x)+w(y))/2$
    \item $w(X) \ge (h+\log_2 r)r2^{h}/c$.
     \item For each internal node $v\in V(T)\cap X$ with children $x$ and $y$,
       $w(x) > 2w(y)$.
  \end{enumerate}
  Now, for each $i\in\{1,\ldots,h\}$, let $X_i\subset X$ denote the
  set of nodes $v\in X$ for which the path from the root of $T$ to $v$
  contains exactly $i$ nodes in $X$ (including $v$).  Observe that $s(X_1)
  \ge s(X_2) \ge \cdots\ge s(X_h)$.

  We will show that, there exists a constant $t>1$ such that, 
  for each $i\in\{1,\ldots,h-t\}$,  $s(X_i+t) < (1/2)s(X_i)$.  In this way, 
  \begin{align*}
     (h+\log_2 r)r2^{h}/c 
        \le s(X) & = \sum_{i=1}^{\infty} s(X_i) \\
           &\le \sum_{i=1}^{\infty} s(X_{\floor{i/t}+1}) \\
           &\le t\sum_{i=0}^{\infty} s(X_{it+1}) \\
           &\le t\sum_{i=0}^{\infty} (1/2)^ii s(X_1) \\
           &\le t\sum_{i=0}^{\infty} (1/2)^i r2^{h} \\
           & = 2tr2^{h} 
  \end{align*}
  which is a contradiction if 
  \[    h+\log_2 r/c \ge 2t \enspace . \]

  For each node $v\in X$, let $R(v)$ denote the unique child of $v$
  such that $w(R(v)) < 2w(v)/3$ (the existence of $R(v)$ follows
  from Properties~3 and 5, above).  Consider some non-empty subset
  $A_0\subseteq X_i$, let $R(A_0)=\{R(x):x\in A_0\}$.  Observe that
  $s(R(A_0)) = s(A_0)/2$ and that $w(R(A_0)) < 2w(A_0)/3$.  Now, let $A_1$
  be the subset of $X_{i+1}$ that are descendants of nodes in $R(A_0)$.
  More generally, we define $A_j$ as the subset of nodes in $X_{i+j}$
  that have ancestors in $R(A_{j-1})$.  The key observation is that
  \begin{equation}
      w(A_j) \le (2/3)w(A_{j-1})\frac{s(A_{j-1})}{s(A_j)} \enspace .
        \eqlabel{string}
  \end{equation}
  Beginning at $A_0$ and repeatedly applying \eqref{string}, we obtain:
  \[
      w(A_t) \le (2/3)^t w(A_0)\frac{s(A_0)}{s(A_t)} \enspace .
  \]
  On the other hand, $w(A_t)\ge 1/r$, so  
  \[ 
      1/r \le (2/3)^t w(A_0)\frac{s(A_0)}{s(A_t)} \enspace .
  \]
  and rearranging terms yields
  \[
     s(A_t) \le (2/3)^t r w(A_0)s(A_0) \le (2/3)^t r s(A_0) \le (1/2)s(A_0)
  \]
  for $t = \lceil\log_{3/2}(2r)\rceil$.
\end{proof}


\begin{thm}
  For every integer $\alpha$, there exists a graph of pathwidth 3 whose
  anagram-free chromatic number is greater than $\alpha$.
\end{thm}

\begin{proof}
Let $G$ be the graph with vertex set
$V(G)=\{v_0,\ldots,v_{n-1},w_0,\ldots,w_{n-1}\}$ and with edge set
\[
  E(G) = \{v_iw_i : i\in\{0,\ldots,n-1\}\} \cup \bigcup_{i=0}^{n-2} \{v_iw_{i+1},v_{i+1}w_i\} \enspace .
\]
We call the pairs $v_i$ and $w_i$ \emph{twins}.
The graph $G$ has pathwidth 3 as can be seen from the path decomposition $B_0,\ldots,B_{n-2}$ where $B_i=\{v_i,w_i,v_{i+1},w_{i+1}\}$.

Now, consider some colouring $\varphi:V(G)\to\{1,\ldots,\alpha\}$ of $G$
using $\alpha$ colours. From this we derive a string $s=s_0\ldots,s_{n-1}$
where $s_i=\{\varphi(v_i),\varphi(w_i)\}$.  Observe that $s$ is
a string over an alphabet $\Sigma$ of size $c=\binom{\alpha}{2}$.
By \lemref{balanced}, $s$ contains a balanced substring $t$ if $n$
is sufficiently large.  Since $t$
is balanced, a na\"ive greedy algorithm allows us to construct a bipartite
 graph $B$ whose edges join
elements of the first half of $t$ to equal elements in the second half
and having the following properties: $B$ has minimum degree 1, maximum
degree 2 and contains no path of length greater than 2.  Thus, the
connected components of $B$ are paths of length 1 and paths of length 2.

We will use $B$ to construct a path $P$ in $G$ whose colouring is an
anagram.  For each length-1 component $s_is_j$ of $B$, the path $P$
will include $v_i$, $w_i$, $v_j$ and $w_j$.  For each length-2 path
$s_is_js_k$ the path $P$ include $v_i$, $v_j$, $w_j$, and the unique
vertex $u\in\{v_k,w_k\}$ such that $\varphi(u)\neq \varphi(v_i)$.
The ordering of vertices along the path $P$ is by increasing order
of index, with ties (among $v_i$ and $w_i$) broken arbitrarily.
The fact that $B$ has minimum degree 1 ensures that $P=x_1,\ldots,x_m$
is indeed a path in $G$.  The choice of vertices in $P$ ensures that
$\varphi(x_1),\ldots,\varphi(x_m)$ is an anagram since length-1 components
of $B$ create twins in the first half that cancel twins in the second
half and length-2 components create twins in one half that cancel two
vertices in the other half.
\end{proof}

\end{document}


